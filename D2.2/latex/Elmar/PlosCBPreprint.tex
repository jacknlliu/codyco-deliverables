

Supportive hand contacts are essential for mastering every-day life tasks, for 
example, when reaching for a glass on the highest shelf humans typically have to 
use the other hand to support their body on the kitchen table. In such 
scenarios, the motion of the body and both arms have to be perfectly 
synchronized to successfully perform the reaching motion and to simultaneously 
ensure the postural stability. However, little is known about the underlying 
processes that govern the motion of the human body during and after the learning 
of these kinds of concurrent motor skills. To study the effect of supportive 
contacts on motor control of reaching, an innovative full-body experimental 
paradigm was established that extends current experimental methods to a more 
ecological setting. The task of the subjects was to reach with their right arm 
for a distant target on a screen while postural stability could only be 
maintained by establishing an additional supportive hand contact with their left 
arm. To examine adaptation, non-trivial postural perturbations of the subjects' 
support base were systematically introduced. A novel probabilistic trajectory 
model approach was employed to analyze the correlation between the motions of 
both arms. We found that subjects adapted to the perturbations by establishing 
supportive hand contacts that were dependent on the location of the reaching 
target. Moreover we found that the trunk motion adapted significantly faster 
than the motion of the arms. However, the most striking finding was that 
observations of the initial phase of the left arm or trunk motion (100-400 ms) 
were sufficient to faithfully predict the complete movement of the right arm. 
Overall, our results suggest that the goal-directed arm movements determine the 
supportive arm motions that ensure postural stability and that adaptation 
happens on different time scales, where the motion of heavy body parts adapts 
faster than light arms. 

The details are in the confidential appendix.

