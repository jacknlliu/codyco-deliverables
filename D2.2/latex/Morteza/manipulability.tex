
This chapter introduces a set of metrics to study, analyse and measure the
ability to balance for both humans and legged robots.  This set of metrics,
which we call manipulability of the center of mass, are defined based on the
concept of end-effector manipulability in the literature.  Regarding the
center of mass as an end-effector and using impulsive dynamics, the metrics
are calculated to study the ability to move and accelerate this point.  They
graphically show the instantaneous change of the center of mass velocity due
to the unit weighted norm of instantaneous changes of the joint velocities or
impulses at the joints.  The proposed metrics can be computed for humans and
general legged robots with floating base and multiple contacts with the
environment in 3D space.  This chapter also provides the results of
experiments on humans to verify the application of the metrics.  In the
experiments, the centers of mass of human subjects in different configurations
are perturbed and joint torques are computed by using inverse dynamics.  The
metrics are shown to be suitable for comparing different postures in the sense
of the total required effort for balance maintenance.

Further details are in the confidential appendix.




