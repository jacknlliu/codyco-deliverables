
\subsection{Introduction} 
High stepping reaction time is a predictor of future falling \cite{Lord2001}, possibly due to inadequate weight shifts preceding foot lift-off \cite{Cohen2011}\cite{Sparto2013}. These can occur due to external balance perturbations \cite{Mille2014} or incorrect planning \cite{Cohen2011}\cite{Sparto2013}. Thus, incorrect weight shift planning might be linked to falls \cite{Robinovitch2013}. To investigate the effect of incorrectly scaled weight shifts on stepping, we developed a novel robotic platform able to amplify subjects’ weight shifts in real-time.

\subsection{Methods}
Eleven young adults (23.7$\pm$4.6 years) stepped as fast and accurate as possible to a target suddenly illuminated in front of one of their legs. On 1/3 of the steps the target jumped shortly before foot liftoff, forcing quick and potentially destabilizing adjustments. This task was performed on a moveable platform in three conditions: platform still (baseline, 60 steps and post-adaptation, 30 steps) or moving (adaptation, 90 steps). When moving, the platform doubled subjects’ mediolateral center of mass movement (COM) in real time. Thus, subjects had to plan a smaller COM movement to generate a weight shift appropriate for stepping to targets.
We calculated stepping errors (distance between the target and the foot at landing), step onset time (time between target onset and foot liftoff) and step execution time (time between liftoff and landing). Overall performance was analyzed using rANOVA (target jump x condition) and differences between the first and last five steps of each condition using paired samples t-tests.

\begin{figure*}[!ht]
	\centering
	\includegraphics[width=0.6\linewidth]{"Zrinka/methods"}
	\caption{Experimental setup. The subject is standing on a movable platform and a lighted area appears on the floor in front of his leg, serving as a stepping cue and target.}
	\label{fig:expSetup}
	\vspace{-4 mm}
\end{figure*}

\subsection{Results}
Target jumps increased step execution times and stepping errors (by 50 ms and 49.5 mm, both p $<$ .01).  Step onsets were delayed in adaptation (by 7 ms), but faster in post-adaptation (by 24 ms). Target jumps delayed step onset by 10 ms in baseline and adaptation (condition x jump interaction, p = .03). Step onsets were faster in the first 5 steps of post- adaptation, compared to last adaptation steps (by 37 ms, p = .01, jump). Stepping errors increased at the start of adaptation (by 19.5 mm, p = .02, no jump), but decreased over time, within adaptation (by 21.8 mm, p = .03, target jump) and post- adaptation (by 7.5 mm, p = .047, no jump).

\begin{figure*}[!ht]
	\centering
    \includegraphics[width=0.6\textwidth]{"Zrinka/step error-all steps"}
	\caption{Stepping errors. Error bars indicate standard deviation. Statistics are given in the text.}
	\label{fig:stepErrorAll}
\end{figure*}

\begin{figure*}[!ht]
	\centering
	\includegraphics[width=0.6\textwidth]{"Zrinka/step onset time-all steps"}
	\caption{Step onset time (time from the cue to step to foot lift-off). Error bars indicate standard deviation. Statistics are given in the text.}
	\label{fig:stepOnsetTimeAll}
\end{figure*}

\begin{figure*}[!ht]
	\centering
	\includegraphics[width=0.6\textwidth]{"Zrinka/step exec time-all steps"}
	\caption{Step execution time (time from step onset to landing). Error bars indicate standard deviation. Statistics are given in the text.}
	\label{fig:stepExecTimeAll}
\end{figure*}

\subsection{Conclusions}
When targets jumped, stepping errors increased, but manipulating balance by platform movements had no effect on stepping accuracy. However, manipulating balance delayed step onsets, in line with the need to adjust weight shifts prior to foot liftoff in this novel environment, and was accompanied by increased stepping errors in the first five steps. Although initially significantly perturbed, subjects quickly adapted to stepping in a novel balance environment. Analyses of weight shifting and adaptation rates of weight shifting and stepping to clarify the underlying mechanisms are pending.