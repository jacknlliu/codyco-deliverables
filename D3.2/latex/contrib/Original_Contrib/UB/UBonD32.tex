%%%%%%%%%%%%%%%%%%%%%%%%%%%%%%%%%%%%%%%%%%%%%%%%%%%%%%%%%%%%%%%%%%%%%%%%%%%%%%%%
%2345678901234567890123456789012345678901234567890123456789012345678901234567890
%        1         2         3         4         5         6         7         8


\newcommand{\EQ}{\!\!\!=\!\!\!}

\newcommand{\Bp}{\mathbf{p}}
\newcommand{\Br}{\mathbf{r}}
\newcommand{\Bf}{\mathbf{f}}
\newcommand{\BJ}{\mathbf{J}}
\newcommand{\Bv}{\mathbf{v}}
\newcommand{\BI}{\mathbf{I}}
\newcommand{\BR}{\mathbf{R}}
\newcommand{\BK}{\mathbf{K}}
\newcommand{\BD}{\mathbf{D}}
\newcommand{\BA}{\mathbf{A}}
\newcommand{\Bb}{\mathbf{b}}
\newcommand{\BM}{\mathbf{M}}
\newcommand{\BC}{\mathbf{C}}
\newcommand{\Bg}{\mathbf{g}}
\newcommand{\BS}{\mathbf{S}}
\newcommand{\Bzero}{\mathbf{0}}
\newcommand{\BN}{\mathbf{N}}
\newcommand{\Bh}{\mathbf{h}}
\newcommand{\BW}{\mathbf{W}}
\newcommand{\Bq}{\mathbf{q}}
\newcommand{\BF}{\mathbf{F}}
\newcommand{\Bn}{\mathbf{n}}

\newcommand{\Bomega}{\boldsymbol{\omega}}
\newcommand{\Btau}{\boldsymbol{\tau}}
\newcommand{\Balpha}{\boldsymbol{\alpha}}
\newcommand{\Bbeta}{\boldsymbol{\beta}}



\documentclass[a4paper, 11pt]{article}



% The following packages can be found on http:\\www.ctan.org
\usepackage{graphics} % for pdf, bitmapped graphics files
\usepackage{epsfig} % for postscript graphics files
\usepackage{mathptmx} % assumes new font selection scheme installed
\usepackage{times} % assumes new font selection scheme installed
\usepackage{amsmath} % assumes amsmath package installed
\usepackage{amssymb}  % assumes amsmath package installed
%\usepackage[cm]{fullpage}
\usepackage[top=1.5in, bottom=1.5in, left=1in, right=1in]{geometry}



\begin{document}

\section{Introduction}
\label{intro}

Balancing for legged robots, which is in fact preventing them from falling
over, has always the highest priority in controlling their motions.  This
problem becomes much more challenging if the supporting surface (i.e.\ usually
the ground) is not ``\emph{rigid}''.  In reality, there is no completely rigid
surface but in practice we can assume a surface to be rigid if it is stiff
enough (i.e. deflection is negligible).  Many of existing legged robots are
able to keep their balance on rigid surfaces but still most of them have
difficulties in dealing with compliant supporting surfaces.  To balance a
legged robot on a soft surface (e. g.  on a thick carpet), the compliance of
the contact has to be considered by the controller, otherwise the robot fails
to balance and falls over.

Compliant contacts between robots and their environments have been studied by
some researchers in the area of humanoids \cite{Bouyarmane&Kheddar11},
grasping \cite{Xydes&Kao99}, animated characters \cite{Jain&Liu11}, etc.
However, there have not been much efforts on balancing legged robots on
compliant surfaces.  Here, we tackle this problem and introduce a
momentum-based balancing controller which takes into account the effects of
non-rigid contacts between the robot and its environment.  A momentum-based
controller, controls both linear momentum and angular momentum about the
center of mass (CoM) of the robot.  This type of controller is first suggested
by Goswami and Kallem \cite{Goswami&Kallem04} and then extensively used in
recent years by other researchers \cite{Azadetal14},
\cite{Azad&Featherstone12}, \cite{Hofmannetal09}, \cite{Lee&Goswami12},
\cite{Macchiettoetal09}.


Here, we propose a balance control strategy for a legged robot which has
multiple rigid and compliant contacts with its environment.  This controller
regulates both linear momentum and angular momentum about the center of mass
of the robot by controlling the contact forces at both rigid and soft contact
surfaces.  Assuming that contact forces at the compliant surfaces are known
(i.e. via force-torque sensors) at the current instant, desired contact forces
at the rigid contacts are calculated in order to provide the required rate of
change of the robot's momentum.  On the other hand, since compliant contact
forces are dependent to deformations (and their rates) in the contact areas,
the change of compliant forces are functions of robot's states and joint
accelerations.  Therefore, by converting the balancing problem to an
optimization problem, the controller computes desired joint torques in order
to provide both desired rigid contact forces (for balance control) and joint
acceleration (to regulate the soft contact forces).

Note that, to implement the proposed method in practice, stiffness and damping
coefficients of the contact model have to be estimated beforehand by using
contact model parameter estimation methods such as \cite{Dallalietal13},
\cite{Diolaitietal05}, \cite{Ericksonetal03}.


\section{Control Strategy}
\label{control}

Let $\Bv$ denote generalized joint velocities of a floating base robot.
Therefore, motion equations for this robot will be
%
\begin{equation}
  \BM \dot{\Bv} + \BC \Bv + \Bg = \BS^T \Btau + \BJ_s^T \Bf_s + \BJ_r^T \Bf_r
  \, ,
  \label{motionEq}
\end{equation}
%
where $\BM$ is the joint-space inertia matrix, $\BC \Bv$ is the vector of
Coriolis and centrifugal forces, $\Bg$ is the vector of the gravity forces,
$\BS$ is the selection matrix for actuated joints, $\Btau$ is the vector of
joint torques, $\BJ_s$ is the Jacobian of the bodies in contact with soft
surfaces, $\Bf_s$ is the vector of the contact forces at the soft surfaces,
$\BJ_r$ is the Jacobian of the bodies in contact with rigid surfaces and
$\Bf_r$ is the vector of the contact forces at the rigid surfaces.  Due to the
rigid contacts, we have
%
\begin{equation}
  \BJ_r \Bv = \Bzero \; \implies \; \dot{\BJ}_r \Bv + \BJ_r \dot{\Bv} = \Bzero
  \, .
\end{equation}
%
Therefore, from equation (\ref{motionEq}) we conclude that
%
\begin{equation}
  \Btau = (\BJ_r \BM^{-1} \BS^T)^{\#}(-\dot{\BJ}_r \Bv + \BJ_r \BM^{-1} \BC
  \Bv + \BJ_r \BM^{-1} \Bg - \BJ_r \BM^{-1} \BJ_s^T \Bf_s - \BJ_r \BM^{-1}
  \BJ_r^T \Bf_r^{des}) + \BN \Btau_0 \, ,
  \label{taucalc}
\end{equation}
%
where superscript $^\#$ denotes the generalized weighted pseudo-inverse and
$\BN$ is the null-space of $\BJ_r \BM^{-1} \BS^T$ as
%
\begin{equation}
  \BN = \BI - (\BJ_r \BM^{-1} \BS^T)^\# \, \BJ_r \BM^{-1} \BS^T \, .
\end{equation}
%

To control the momentum of the robot for its balancing motion, desired force
at the rigid contact can be calculated by using
%
\begin{equation}
  \Bf_r^{des} = \BA_r^{\#} (\dot{\Bh}^{des} - \BW - \BA_s \Bf_s) + \BN_{A_r}
  \Bf_{r_0} \, ,
  \label{frdes}
\end{equation}
%
where $\BW = [0,0,-9.81,0,0,0]^T$ is the force vector due to the weight of the
robot, $\Bh$ is the vector of linear and angular momentum of the robot about
its center of mass (CoM), and $\BA_s$ and $\BA_r$ are the matrices that
transform the soft and rigid contact forces from their local coordinate frames
to the forces and moments around the CoM, respectively.  Also $\BN_{A_r} = \BI
- \BA_r^\# \, \BA_r$ is the null-space of $\BA_r$.  Here, $\Bf_{r_0}$ can be
used to satisfy the constraints (i.e. unilaterality of the normal force and
friction cone) on the rigid contact forces.

The null-space term in equation (\ref{taucalc}) is available to be used for
task space control (or impedance control) and also for controlling the soft
contact force.  For task control we have
%
\begin{equation}
  \Btau_t = \BK_q (\Bq_{des} - \Bq) + \BK_{qd} (\dot{\Bq}_{des} - \dot{\Bq}) +
  \ddot{\Bq}_{des} \, ,
  \label{tauImp}
\end{equation}
%
where $\Bq$ is the vector of joint angles and $\BK_q$ and $\BK_{qd}$ are the
gains of the controller.

Since compliant contact forces are functions of surface deformations, we do
not have any control on them at the current instant and they are usually given
via force-torque sensors.  However, it is possible to control those forces in
the next instant by controlling the acceleration of the contact points.  Here,
we assume a linear relationship between soft contact forces at the next
instant (i.e. $\hat{\Bf}_s$) and generalized accelerations as
%
\begin{equation}
  \hat{\Bf}_s = \Bf_s + \BA \dot{\Bv} + \Bb \, ,
  \label{fs_hat}
\end{equation}
%
where $\BA$ and $\Bb$ are described in the next section.  To find a
relationship between $\dot{\Bv}$ and $\Btau$, we first solve equation
(\ref{motionEq}) for $\Bf_r$ as
%
\begin{equation}
  \Bf_r = (\BJ_r \BM^{-1} \BJ_r^T)^{-1} (-\dot{\BJ}_r \Bv + \BJ_r \BM^{-1} \BC
  \Bv + \BJ_r \BM^{-1} \Bg - \BJ_r \BM^{-1} \BJ_s^T \Bf_s - \BJ_r \BM^{-1}
  \BS^T \Btau) \, .
  \label{fr}
\end{equation}
%
By defining
%
\begin{equation}
  (\BJ_r \BM^{-1})^{\#} = \BJ_r^T (\BJ_r \BM^{-1} \BJ_r^T)^{-1} \, ,
\end{equation}
%
and replacing (\ref{fr}) into (\ref{motionEq}), we will have
%
\begin{equation}
  \dot{\Bv} = \BM^{-1} \BN_{JM} (-\BC \Bv - \Bg + \BS^T \Btau + \BJ_s^T \Bf_s)
  - \BM^{-1} (\BJ_r \BM^{-1})^{\#} \dot{\BJ}_r \Bv \, ,
\end{equation}
%
where
%
\begin{equation}
  \BN_{JM} = \BI - (\BJ_r \BM^{-1})^{\#} \BJ_r \BM^{-1} \, .
\end{equation}
%
Hence, $\dot{\Bv}$ is linearly dependent on $\Btau$ and we can write
%
\begin{equation}
  \dot{\Bv} = \Balpha \Btau + \Bbeta \, ,
  \label{vdot_tau}
\end{equation}
%
where
%
\begin{equation}
  \Balpha = \BM^{-1} \BN_{JM} \BS^T \Btau \, ,
\end{equation}
%
and
%
\begin{equation}
  \Bbeta = \BM^{-1} \BN_{JM} (-\BC \Bv - \Bg + \BJ_s^T \Bf_s) - \BM^{-1}
  (\BJ_r \BM^{-1})^{\#} \dot{\BJ}_r \Bv \, .
\end{equation}
%

Therefore, by replacing equation (\ref{vdot_tau}) into (\ref{fs_hat}) we have
%
\begin{equation}
  \hat{\Bf}_s = (\BA \Balpha) \Btau + (\Bf_s + \Bb + \BA \Bbeta) \, ,
\end{equation}
%
which is a linear relationship between $\hat{\Bf}_s$ and $\Btau$.

Now, we can describe the balancing problem as an optimization problem which
its objective is to find $\Btau$ in order to minimize the following function
as
%
\begin{equation}
  \Btau = ArgMin ( w_1 ||\Btau_0 - \Btau_t|| + w_2||\hat{\Bf}_s -
  \Bf_s^{des}||)
\end{equation}
%
where $\Bf_s^{des}$ is the desired forces for the soft contacts and $w_1$ and
$w_2$ are the values representing the priority of each term.  By setting
$\Bf_s^{des}$ we can determine how much we want to rely on the soft contacts
during a balancing motion.  This optimization problem is subject to the
constraints on the saturation limits of the actuators and also the constraints
on the soft contact forces (friction cones and unilaterality of the normal
forces).  Note that contact force constraints for the rigid contacts should be
already considered in computing $\Bf_r$ from (\ref{frdes}).

The next section, explains the estimation of the soft contact forces to obtain
(\ref{fs_hat}), and also constraints on $\hat{\Bf}_s$ that should be
considered in the above optimization problem.


\section{Estimation of the soft contact force}

Assume that the body which is in contact with a soft surface is labeled with
$B_c$.  We characterize the contact surface of $B_c$ by $m$ fictitious contact
points on this body.  Let $\Bp_i$ denote the position of the i$^{th}$ contact
point ($i=1,2,\ldots,m$) in the world frame.  Therefore,
%
\begin{equation}
  \Bp_i = \Bp + \BR \Br_i \, ,
\end{equation}
%
where $\Bp$ is the position of the origin of the local frame of $B_c$ with
respect to the world frame, $\BR$ is the rotation matrix of $B_c$ with respect
to the world frame and $\Br_i$ is the position of i$^{th}$ contact point in
the local frame of $B_c$.  So
%
\begin{equation}
  \dot{\Bp}_i = \dot{\Bp} + \dot{\BR} \Br_i = \dot{\Bp} - (\BR\Br_i)^{\wedge}
  \Bomega = [ \BI_{3 \times 3} \; \; {}-(\BR\Br_i)^{\wedge} ] \BJ_s \Bv \, ,
  \label{pidot}
\end{equation}
%
where $\Bomega$ is the angular velocity of $B_c$ and $()^{\wedge}$ represents
the skew symmetric matrix.  We also have
%
\begin{equation}
  \ddot{\Bp}_i = \ddot{\Bp} + \ddot{\BR} \Br_i = \ddot{\Bp} - (\BR
  \Br_i)^{\wedge} \dot{\Bomega} + (\Bomega)^{\wedge} (\Bomega)^{\wedge} \BR
  \Br_i = [ \BI_{3 \times 3} \; \; {}-(\BR \Br_i)^{\wedge}] (\dot{\BJ}_s \Bv +
  \BJ_s \dot{\Bv}) + (\Bomega)^{\wedge} (\Bomega)^{\wedge} \BR \Br_i \, .
  \label{piddot}
\end{equation}
%

According to contact mechanics \cite{Johnson77} and its applications in
robotics \cite{Azad&Featherstone10, Azad&Featherstone14a, Hunt&Crossley75,
  Marhefka&Orin99}, there is a non-linear relationship between compliant
contact force and deformation and the rate of the deformation of the surface.
However, we can assume a locally linear relationship between the change of the
contact force and the change of the deformation and its rate.  Therefore,
according to this assumption, we can write
%
\begin{equation}
  \delta \Bf_i = \BK \delta \Bp_i + \BD \delta \dot{\Bp}_i \, ,
\end{equation}
%
where $\delta \Bf_i$ and $\delta \Bp_i$ are the changes of the contact force
and the deformation at the i$^{th}$ contact point, respectively, and $\BK$ and
$\BD$ are $3 \times 3$ matrices of the coefficients of stiffness and damping,
respectively.  By using a linear integration method, we can estimate $\delta
\Bp_i$ and $\delta \dot{\Bp}_i$ as
%
\begin{equation}
  \delta \Bp_i = \dot{\Bp}_i \delta t + \frac{1}{2} \ddot{\Bp}_i \delta t^2 \,
  ,
  \label{deltapi}
\end{equation}
%
and
%
\begin{equation}
  \delta \dot{\Bp}_i = \ddot{\Bp}_i \delta t \, ,
  \label{deltapidot}
\end{equation}
%
where $\delta t$ is the sampling time.  Hence, by substituting (\ref{pidot})
and (\ref{piddot}) into (\ref{deltapi}) and (\ref{deltapidot}), we will have
%
\begin{equation}
  \delta \Bf_i = \BA_i \dot{\Bv} + \Bb_i \, ,
\end{equation}
%
where
%
%
\begin{equation}
  \BA_i = \frac{1}{2} \delta t^2 [\BK \; \; -\BK(\BR \Br_i)^{\wedge}] \BJ_s +
  \delta t [\BD \; \; - \BD(\BR \Br_i)^{\wedge}] \BJ_s \, ,
\end{equation}
%
and
%
\begin{eqnarray}
  \nonumber \Bb_i & \EQ & \delta t [\BK \; \; -\BK(\BR \Br_i)^{\wedge}] \BJ_s
  \Bv + \frac{1}{2} \delta t^2 [\BK \; \; -\BK(\BR \Br_i)^{\wedge}]
  \dot{\BJ}_s \Bv\\ & & + \frac{1}{2} \delta t^2 \BK (\Bomega)^{\wedge}
  (\Bomega)^{\wedge} \BR \Br_i + \delta t [\BD \; \; -\BD(\BR \Br_i)^{\wedge}]
  \dot{\BJ}_s \Bv + \delta t \BD (\Bomega)^{\wedge} (\Bomega)^{\wedge} \BR
  \Br_i \, .
\end{eqnarray}
%
Also for the moment of the contact forces we have
%
\begin{equation}
  \delta \Bn_i = (\BR \Br_i)^{\wedge} \delta \Bf_i \, .
\end{equation}
%
Therefore, total change of force vector for $B_c$ will be
%
\begin{equation}
  \delta \Bf_s =
  \begin{bmatrix}
    \sum_{i=1}^m \BA_i \dot{\Bv} + \sum_{i=1}^m \Bb_i\\
    \\
    \sum_{i=1}^m (\BR \Br_i)^{\wedge} \BA_i \dot{\Bv} + \sum_{i=1}^m \Bb_i
  \end{bmatrix}
  =
  \begin{bmatrix}
    \BA_f \\
    \\
    \BA_n
  \end{bmatrix}
  \dot{\Bv} +
  \begin{bmatrix}
    \Bb_f \\
    \\
    \Bb_n
  \end{bmatrix}
  = \BA \dot{\Bv} + \Bb \, .
\end{equation}
%
Hence, the estimated (for the next instant) 6D force-torque vector of the
contact force of $B_c$ will be
%
\begin{equation}
  \hat{\Bf}_s = \Bf_s + \delta \Bf_s = \BA \dot{\Bv} + \Bb + \Bf_s \, ,
\end{equation}
%
which is the same as (\ref{fs_hat}) that we already used in the control
algorithm.  Note that for the constraints on the soft contact force
(unilaterality and friction cone), $\hat{\Bf}_s$ has to be expressed in the
local frame of $B_c$ as $\hat{\BF}_s = \BR^T \hat{\Bf}_s$.



\begin{thebibliography}{16}

\bibitem{Azadetal14} M. Azad, J. Babi\v{c} and M. Mistry, Effects of hand
  contact on the stability of a planar humanoid with a momentum based
  controller, Proc. IEEE-RAS Int. Conf. Humanoid Robots, Madrid, Spain, 18--20
  Nov 2014.
  
\bibitem{Azad&Featherstone10} M. Azad and R. Featherstone, Modeling the
  contact between a rolling sphere and a compliant ground
  plane. Proc. Australasian Conf. Robotics and Automation, Brisbane,
  Australia, 1--3 Dec 2010.

\bibitem{Azad&Featherstone12} M. Azad and R. Featherstone, Angular momentum
  based controller for balancing an inverted double pendulum. RoManSy 19-Robot
  Design, Dynamics and Control, pp. 251--258, Paris, France, 12--15 June 2012.

\bibitem{Azad&Featherstone14a} M. Azad and R. Featherstone, A new nonlinear
  model of contact normal force, IEEE Trans. Robotics, 30(3):736--739, June
  2014.

\bibitem{Bouyarmane&Kheddar11} K. Bouyarmane and A. Kheddar, FEM-based static
  posture planning for a humanoid on deformable contact support, IEEE-RAS
  Int. Conf. Humanoid Robots, pp. 487--492, Bled, Slovenia, 26--28 Oct 2011.

\bibitem{Dallalietal13} H. Dallali, M. Mosadeghzad, G. A. Medrano-Cerda and
  N. Docquier, Development of a dynamic simulator for a compliant humanoid
  robot based on a symbolic multibody approach, IEEE Int. Conf. Mechatronics,
  pp. 598--603, Vicenza, Italy, Feb 2013.

\bibitem{Diolaitietal05} N. Diolaiti, C. Melchiorri, S. Stramigioli, Contact
  impedance estimation for robotic systems, IEEE Trans. Robotics,
  21(5):925--935, Oct 2005.

\bibitem{Ericksonetal03} D. Erickson, M. Weber, I. Sharf, Contact stiffness
  and damping estimation for robotic systems, Int. J. Robotics Research,
  22(1):41--57, Jan 2003.

\bibitem{Goswami&Kallem04} A. Goswami and V. Kallem, Rate of change of angular
  momentum and balance maintenance of biped robots, Proc. IEEE
  Int. Conf. Robotics and Automation, pp. 3785--3790, New Orleans, LA, April
  2004.

\bibitem{Hofmannetal09} A. Hofmann, M. Popovic and H. Herr, Exploiting angular
  momentum to enhance bipedal center-of-mass control, IEEE Int. Conf. Robotics
  and Automation, pp.4423--4429, Kobe, Japan, 12--17 May 2009.

\bibitem{Hunt&Crossley75} K. H. Hunt and F. R. E. Crossley, Coefficient of
  restitution interpreted as damping in vibroimpact, J. Applied Mechanics,
  42(2):440--445, 1975.

\bibitem{Jain&Liu11} S. Jain and C. K. Liu, Controlling physics-based
  characters using soft contacts, ACM Trans. Graphics, 30(6):163, 2011.

\bibitem{Johnson77} K. L. Johnson, Contact mechanics, Cambridge University
  Press, 1977.
  
\bibitem{Lee&Goswami12} S. H. Lee and A. Goswami, A momentum-based balance
  controller for humanoid robots on non-level and non-stationary ground,
  Autonomous Robots, 33(4):399--414, Springer, 2012.

\bibitem{Macchiettoetal09} A. Macchietto, V. Zordan and C. R. Shelton,
  Momentum control for balance, ACM Trans. Graphics, 28(3):80--87, 2009.

\bibitem{Marhefka&Orin99} D. W. Marhefka and D. E. Orin, A compliant contact
  model with nonlinear damping for simulation of robotic systems, IEEE
  Trans. Systems, Man, and Cybernetics-part A: Systems and Humans,
  29(6):566--572, 1999.

\bibitem{Xydes&Kao99} N. Xydes and I. Kao, Modeling of contact mechanics and
  friction limit surfaces for soft fingers in robotics, with experimental
  results, Int. J. Robotics Research, 18(9):941--950, Sep 1999.

\end{thebibliography}

\end{document}
