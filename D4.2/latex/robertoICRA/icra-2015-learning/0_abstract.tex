% %
% %% General problem
% Controlling a robot in the presence of contacts with the environment is an open research problem. 
% In particular, one challenge needs to be addressed: the contact torques\slash forces need to be estimated.
% %
% %% Detailed problem
% % Why no torque sensors.
% % PROBLEM!!! We also need them!!!!
% In principle, the contact forces can be directly estimated with force/torque sensors. 
% However, they are expensive and cannot be integrated in all joints. 
% Additionally, the exact position of the contact \fixme{plus other stuff} must be known precisely.
% % Why no skin
% A different approach is to use tactile sensing to identify the contact forces, in conjunction with an analytic model of the contact.
% %Another approach is to use analytic models of contacts. 
% This requires the position of the sensors on the body to be precisely known. 
% Moreover, designing a model becomes complicated with an increasing number of sensors.  
% Therefore, often only an approximate contact model is known. 
% 
% %% Our contribution
% In this paper, we investigate how contact forces can accurately be learned using raw tactile sensing. 
% For this purpose, we propose a novel approach to learning a contact force model directly from data using non-parametric regression methods. 
% 
% %% Our experiments
% As a proof of concept, we validate our approach on a humanoid robot \robot{}. 
% We first show that the learned model can accurately predict the contacts force of a single contact exclusively using tactile information.
% Our approach extends to multiple contacts, and we show that it outperforms existing contact models that make use of force\slash torque sensors in situations with multiple simultaneous contacts.
%
%
%
%
%
%In many settings, robots interact with their %environment and manipulate objects.
%% General problem
%Contacts pose key research challenges when modeling and controlling a robot that interacts with its environment.
%Contacts have non-linear effects on the system dynamics that are difficult to model accurately and may cause interaction forces that could harm a robot.
%
%Therefore, for adaptive control strategies the contact torques\slash forces need to be estimated accurately. In principle this can be done through joint-torque sensors.
In whole-body control, joint torques and external forces need to be estimated accurately. In principle, this can be done through pervasive joint-torque sensing and accurate system identification.
However, these sensors are expensive and may not be integrated in all links.
Moreover, the exact position of the contact must be known for  a precise estimation. 
If contacts occur on the whole body, tactile sensors can estimate the contact location, but this requires a kinematic spatial calibration, which is prone to errors. 
Accumulating errors may have dramatic effects on the system identification.
%
% REMOVE THIS?? ==>
% However, these sensors are expensive, may not be integrated in all links, and the exact position of the contact must be known. Already small deviations from the true contact location might have dramatic effects on the estimated dynamics. 
% Alternatively, tactile sensing may be used to estimate the contact location and the forces, as in existing analytically modeling approaches. Also in these methods the position of each sensor element (typical sensor arrays consist of 100--1000 elements) has to be precisely known.
%
% <== REMOVE THIS??
%Existing approaches model contacts analytically, where the position of each sensor element in the robot body (typical sensor arrays consist of 100--1000 elements) has to be precisely known.
%
% Alternatively, we can use tactile sensing to estimate the contact forces.
% % <== REMOVE THIS??
% Existing approaches model contacts analytically, where the position of each sensor element in the robot body (typical sensor arrays consist of 100--1000 elements) has to be precisely known.
% %% Our contribution
%
%
As an alternative to classical model-based approaches we propose a data-driven mixture-of-experts learning approach using Gaussian processes. 
This model predicts joint torques directly from raw data of tactile and force\slash torque sensors.
%In particular, a mixture-of-experts approach based on Gaussian processes is used to learn a mapping from tactile sensor inputs to observed contact forces.
%% Our experiments
%
We compare our approach to an analytical model-based approach on real world data recorded from the humanoid \robot{}.
We show that the learned model accurately predicts the joint torques resulting from contact forces, is robust to changes in the environment and outperforms existing dynamic models that use of force\slash torque sensor data.