% READ: http://www.ausy.tu-darmstadt.de/HowTo/WritingAnEffectiveAbstract
%
% To deal with contacts in whole-body control it is necessary to estimate 
% accurately joint torques and external forces.
%
%
%
%
% General problem statement
%
Whole-body control in presence of unknown obstacles is a challenging task.
%
% JP: R ask Tucker on obstacles vs obstructions
%
Unforeseen contacts with such obstacles can lead to poor tracking performance and potential physical damages.
Hence, a whole-body control approach for future humanoid robots in unmodeled environments needs to take contact sensing into account. 
However, converting contact sensed with skin into physically well-understood quantities can be problematic as the exact position and strength of the contact would have to be converted into torque. 

%
% What is our paper about?
In this paper, we suggest an alternative approach that directly learns the mapping from both skin and joint state to the required torques needed for controlling the desired trajectory. 
%
%
% How do we accomplish this? 
We propose to learn such an inverse dynamics models with contacts using a ``mixture of contacts'' approach that exploits the linear superimposibility of contact forces. 
The learned model can accurately predict torques needed to compensate for the contact.  
%
%
% What's new or better?  
As a result, trajectories with tactile contact can be executed more accurately even with low feedback gains and reduced risk of physical damage to both robot and environment. 

%
% What's the evidence of the advantages?
We demonstrate on two different tasks on the humanoid robot~\robot{} that this controller has a lower tracking error than classical alternatives.


% From the old one
%Therefore, joint torques and external forces due to contacts need to be estimated accurately. 
%In principle, this can be done through pervasive joint-torque sensing with accurate dynamics models, requiring identification of the dynamics parameters.
%However, these sensors are expensive, may not be integrated in all links, and the  must be known. 
%If contacts occur on the whole-body, tactile sensors can be used to estimate the contact location, but this requires a kinematic spatial calibration, prone to errors. 
%Accumulating errors may have dramatic effects on the estimated dynamics.
%
%% Our contribution
%In this paper, we demonstrate that it is possible to learn  to accurately model the effect of contacts by using tactile sensors.

