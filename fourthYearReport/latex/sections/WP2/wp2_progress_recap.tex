%!TEX root = ../../fourthYearReport.tex
 
\paragraph*{WP2: understanding and modelling human whole-body behaviours in physical interaction (JSI)}

In T2.2 and also T2.3, UB continued on improving CoM dynamic manipulability as a tool to study, analyze and measure physical abilities of humans and robots.

In T2.4 UB explored the mechanism of human force perception aiming to provide natural and stable control for a humanoid robot in the similar manner with humans. Through this work, an experimental design has been developed, and a series of human subject experiments examined the anticipated goal-directed behaviour interacting with different compliant force dynamics. They also focused on compliant contacts with support surfaces under uncertainty, where one of the important issues is the extraction of information about the contact surfaces through the sense of touch. UB conducted two psychological studies where they examined tactile roughness through perceptual judgments and brain activation, and one study where they examined the effects of tangential load force uncertainty on precision grip cooperative lifting. Further to these studies, UB developed a preliminary study which was performed using the Haptic master along with a simulated enviroment. 

In T2.4 UPMC analyzed social and physical signals in human-robot (iCub) interaction during a collaborative assembly task. The experiments, performed by Serena Ivaldi at UPMC, were later analyzed during her time in TUD and INRIA.

With a follow-up of the previous work in T2.2, which showed how contacts are established to facilitate goal directed movements, JSI performed a study to answer an inverse question: how would a contact be released? Furthermore (in line with T2.4), if there is an uncertainty of balance, how does this contact preference change?

On their newley developed real-time movable platform perturbation system, JSI investigated effects on balance during quiet standing. A memory task and an inhibitory Stroop task were performed to explore cognitive control over balance in novel challenging conditions.

\begin{itemize}

\item 
\item 
\item 
\item 

 \end{itemize}
