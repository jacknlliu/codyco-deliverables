%!TEX root = ../../fourthYearReport.tex


 
\paragraph{Work package 3 progress}

The progress for each task are described hereafter.

\subparagraph{Reproducing existing control results in a simple case (T3.1)}

The explicit goal of T3.1 for the fourth year was to $\dots$

We achieved the following results $\dots$

\subparagraph{Formulating the control problem (T3.2)}

The explicit goal of T3.2 for the fourth year was to $\dots$

We achieved the following results $\dots$

\subparagraph{Solving the local control problem (T3.3)}

During year 4, IIT proposed a control laws ensuring the stabilization of a time-varying desired joint trajectory and joint limit avoidance (see Fig.~\ref{fig:icub leg limits} for an illustration of the limits on the leg) in the case of fully-actuated manipulators. The key idea is to perform a parametrization of the feasible joint motion space in terms of exogenous states $\xi$, in the form of $q(\xi) := \delta \tanh(\xi) + q_0$, where $q$ is the joint position, $\delta$ the range of feasible motion and $q_0$ its middle value. It follows that the control of the exogenous states allows for joint limit avoidance. One of the main outcomes of this work is that position terms in control laws are replaced by parametrized terms. Stability and convergence of time-varying reference trajectories obtained with the proposed method were demonstrated to be in the sense of Lyapunov. The introduced control laws were verified by carrying out experiments on two degrees-of-freedom of the torque-controlled iCub. This work led to a publication in Humanoid s 2016 \cite{charbonneau2016Humanoids}.

\begin{figure*}
   \begin{center}
    \includegraphics[width=0.4\textwidth]{iCub_joint_limits_leg.pdf}
    \caption{iCub leg setup used for the experiments. The red circles identify the hip and knee joints, while the white marks indicate joint limits. The green arrow shows the external force applied during experiments.}
    \label{fig:icub leg limits}
    \end{center}
\end{figure*}    

\subparagraph{Bootstrapping and validating the control approach in rigid world and compliant cases (T3.4)}

TUD and Inria, in collaboration with WP4, continued the collaboration on the topic of learning torque control in presence of multiple contacts, exploiting the force/torque and tactile sensors of iCub. Machine learning techniques were used to directly learn the mapping from both skin and the joint state to torques, using mixtures of contact models. Recently, the model was improved for torque control by addressing critical issues in learning from high-dimensional inputs, such as the artificial skin. It was demonstrated that it is possible to considerably reduce the dimensionality of the skin data preserving the information content of the contact position by using stacked auto encoders. A journal paper is currently in preparation. This technique will allow improving torque control in presence of multiple contacts (rigid and/or soft). This work is part of the PhD thesis work of Roberto Calandra on ``Bayesian Modeling for Optimization and Control in Robotics'' \cite{calandra2016PhD}.\\

Inria also designed a multi-task prioritized controller with soft task priorities. The controller was first designed in \cite{modugno2016ICRA} and applied to classical manipulators. In \cite{modugno2016Humanoids} it was extended for whole-body movements. In the latter work, it was used to generate safe whole-body behaviors for iCub, reaching multiple goals and avoiding obstacles, with the guarantee that the generated behaviors were not violating the constraints of the platform. A software controller for iCub has been prototyped in Matlab then in C++.
    
\subparagraph{Deviations from workplan}  

The PM expenses for WP3 after one year of project are globally conform to the planned one. The observed deviations are related to the fact that tasks 3.3 and 3.4 spans the overall duration of the project and the contribution of some of the partners are expected in the 2nd, 3rd and 4th year.

%\emph{\color{red}[For work package 3 (UPMC) provide the following information:]}
%\begin{itemize}
%\item[-] \emph{\color{red}[A summary of progress towards objectives and details for each task;]}
%\item[-] \emph{\color{red}[Highlight clearly significant results;]}
%\item[-] \emph{\color{red}[If applicable, explain the reasons for deviations from Annex I and their impact on other tasks as well as on available resources and planning;]}
%\item[-] \emph{\color{red}[If applicable, explain the reasons for failing to achieve critical objectives and/or not being on schedule and explain the impact on other tasks as well as on available resources and planning (the explanations should be consistent with the declaration by the project coordinator) ;]}
%\item[-] \emph{\color{red}[a statement on the use of resources, in particular highlighting and explaining deviations between actual and planned  person-months per work package and per beneficiary in Annex 1 (Description of Work);]}
%\item[-] \emph{\color{red}[If applicable, propose corrective actions.]}
%\end{itemize}




