%% Template for EU deliverable, using the deliverable.sty style file

\documentclass[12pt,a4paper,twoside]{article}

%% common package
\usepackage[headers]{deliverable}
\usepackage{xspace}
\usepackage{verbatim}
\usepackage[usenames]{color}
\usepackage[usenames,dvipsnames]{xcolor}
\usepackage[dvips]{graphicx}
\usepackage{url}
\usepackage{array}
%%

%%insert here other packages needed by sections

%%

%%%%%%%%%%%%%%%%%%%%%%%%%%%%%%%%%%%%%%%%%%%%%%%%%%%%%%%%%%%%%%%%%%%%%%%%%%%%%%
%%% Titlepage
%%%%%%%%%%%%%%%%%%%%%%%%%%%%%%%%%%%%%%%%%%%%%%%%%%%%%%%%%%%%%%%%%%%%%%%%%%%%%%

% declaration of variables used in style
\deliverableDocnumber{D7.1}
\deliverableTitle{Dissemination and exploitation plan.}

\deliverableAuthor{Francesco Nori}
\deliverableResponsiblePartner{IIT}
\deliverableAffiliation{% Insert here authors affiliations
 $^1$ IIT
}

\deliverableReviewer{Francesco Nori}
\deliverableCoordinator{Francesco Nori}
\deliverableActivityNumber{n} %% n=1,..,10
\deliverableActivity{RTD}
\deliverableDoctype{Deliverable} %% or Prototype
\deliverableClassification{Public} % or Consortium
\deliverableDistribution{Consortium} %
\deliverableStatus{Draft} % Draft or Final
\deliverableDeliveryDate{28/2/2014}
\deliverableFile{D7.1.pdf} % please do not use "-" in the name
\deliverableVersion{1.0}
\deliverableDate{Feb.~28, 2014}
\deliverableYear{2014}
\deliverablePages{\pageref{LastPage}}
\deliverableChangelog{v.1.0 & Feb 19, 2013 & First draft %%\\\hline
%%              v.2.0 & Feb 20, 2007 & Final version
}
\deliverableProjectStartingDate{1st March 2013}
\deliverableProjectEndDate{28th February 2017}
\deliverableProjectAcronym{CoDyCo}
\deliverableProjectTitle{Whole-Body Compliant Dynamical Contacts in Cognitive Humanoids}
 \deliverableContractNumber{600716}
 \deliverableProjectCoordinator{Istituto Italiano di Tecnologia}
 \deliverableProjectUrl{www.codyco.eu}
 \deliverableFrameworkProgramme{FP7}
 
 \deliverableWorkpackage{deliv WP7}
 \deliverableEditors{Francesco Nori and Francesca Boscolo}
 \deliverableContributors{Francesco Nori and Francesca Boscolo}
 \deliverableReviewers{}
\deliverableAbstract{In this document we provide a comprehensive exploitation plan }
\deliverableReviewers{}
\deliverableKeywordList{Exploitation, results, transfer, intellectual properties, promotion.}

%%%%%%%%%%%%%%%%%%%%%%%%%%%%%%%%%%%%%%%%%%%%%%%%%%%%%%%%%%%%%%%%%%%%%%%%%%%%%%
%%% Sections
%%%%%%%%%%%%%%%%%%%%%%%%%%%%%%%%%%%%%%%%%%%%%%%%%%%%%%%%%%%%%%%%%%%%%%%%%%%%%%

%% constants
\newcommand{\botegoCaps}{BOTEGO}
\newcommand{\certhCaps}{CERTH}
\newcommand{\cybionCaps}{CYBION}
\newcommand{\nuigCaps}{NUIG}
\newcommand{\ubitechCaps}{UBITECH}

%%
%%%%%%%%%%%%%%%%%%%%%%%%%%%%%% BEGIN DOCUMENT
\begin{document}

\deliverableMaketitle

%%TODO move to style
\newcolumntype{L}[1]{>{\raggedright\let\newline\\\arraybackslash\hspace{0pt}}m{#1}}
\newcolumntype{C}[1]{>{\centering\let\newline\\\arraybackslash\hspace{0pt}}m{#1}}
\newcolumntype{R}[1]{>{\raggedleft\let\newline\\\arraybackslash\hspace{0pt}}m{#1}}

\textbf{Document Revision History}
\begin{center}
\begin{tabular}{|C{2cm}|C{3cm}|p{5cm}|C{4cm}|}
\hline
\textbf{Version}&\textbf{Date}&\textbf{Description}&\textbf{Author}\\\hline
First draft & 19 Feb 2014 & This version mainly contains the information that was collected via email from the partners. Partners provided IIT with details on their technological transfer facilities. & Francesco Nori\\\hline
\end{tabular}
\end{center}
 
 \clearpage

\newpage
\renewcommand*\contentsname{Table of Contents}
\renewcommand*\listfigurename{Index of Figures}
\tableofcontents
\newpage
% \listoffigures
\newpage

%%%%%%%%%%%%%%%%%%%%%%%% Start deliverable content here.

\section{Introduction}

\section{Executive Summary}

\section{Technological transfer facilities}
In this section we describe the technological transfer facilities per each of the CoDyCo partners. We discuss in particular relevant information such as number of people involved in the facilities, a brief description of the activities and a tentative list of companies which collaborate with the institution. 

\subsection{The IIT technological transfer facility}

\subsubsection{Name of the office for tech transfer (or the person in charge)}
\subsubsection{Number of people involved in the activity (possibly with contact details)}
\subsubsection{Rough estimate of the amount of projects managed by the office (e.g. patents, spin-off, etc.)}
\subsubsection{Brief description of the activities of the tech transfer office}
\subsubsection{Initiatives of the tech transfer office: competitions, awards, etc.}
\subsubsection{List of companies which collaborate with your institution}

\subsection{The TUD technological transfer facility}

\subsubsection{Name of the office for tech transfer}
Tech transfer services are provided by two units at TU Darmstadt:
\begin{itemize}
\item Tech Transfer Office (Referat VI E: Transfer), headed by Dr. Annette Miller-Suermann.
\item Industry Liaison Office (Referat VI B: Kooperationen), headed by Dr.-Ing. Nicolas Repp.
\end{itemize}

\subsubsection{Number of people involved in the activity}
\begin{itemize}
\item Tech Transfer Office: 7 members of staff, including HIGEHST (Home of Innovation, Growth, Entrepreneurship and Technology Management)
\item Industry Liaison Office : 5 members of staff
Contact details can be found here: 
\url{http://www.intern.tu-darmstadt.de/dez_vi/ansprechpartnerinnen/ansprechpartner.de.jsp}
\end{itemize}

\subsubsection{Rough estimate of the amount of projects managed by the office}
\begin{center}
\begin{tabular}{| l | c | c | c | c | r | }
\hline
Patents & 2010 & 2011 & 2012 & 2013 \\
\hline
Invention disclosures & 72 & 87 & 68 & 73\\
Patent applications & 33 & 39 & 45 & 18\\ 	 	 	 	 
Currently active property rights (national) & 46 & 67 & 79 & 88\\
Currently active property rights (international)& 37 & 43 & 79 & 94\\
\hline
\bf{Total amount} & 83 & 110 & 158 & 182 \\
\hline
\end{tabular}
\end{center}

Spin-offs:
\begin{itemize}
\item From 2010 to 2012: 29 spin-offs founded.
\item Since 2013: 12 spin-offs founded.
\item In 2013 more than 166 prospective entrepreneurs got initial advice by the Tech Transfer Office / HIGHEST.
\end{itemize}

Examples of TU Darmstadt spin-offs can be found here: 
\url{http://www.highest.tu-darmstadt.de/highest/gruendungsbeispiele/index.de.jsp}

 
\subsubsection{Brief description of the activities of the tech transfer office}
\begin{itemize}
\item Tech Transfer Office: information, advice and support for the following topics: IP management, commercialization of research results, entrepreneurship (HIGHEST).
\item Industry Liaison Office : information, advice and support w.r.t. industry liaison topics (e.g. matchmaking, first level contractual advice), management of private public partnerships on a strategic level, key account management for industry partners, organization of trade fair participations.
\end{itemize}

\subsubsection{Initiatives of the tech transfer office}
The Tech Transfer Office / HIGHEST is organizing an annual “ideas competition” (“TU Darmstadt Ideenwettbewerb”) for students as well as members of staff.

\subsubsection{List of companies which collaborate with the institution}
TU Darmstadt is collaborating with a wide range of companies of all sizes. On a strategic level, the following companies are strongly connected to TU Darmstadt (e.g.in form of joint research labs or strategic partnerships):
\begin{itemize}
\item Deutsche Bahn
\item Continental
\item Merck
\item Intel
\item SAP
\end{itemize}
Unfortunately, we cannot provide a full list of our collaborations due to nondisclosure agreements with our partners.


\subsection{The UPMC technological transfer facility}

\subsubsection{Name of the office for tech transfer}
SATT LUTECH

\subsubsection{Number of people involved in the activity}
Research potential: more than 7,500 FTE (Full-time equivalent) scientists and research staff. A potential equivalent to UC Berkeley, Univ. Wisconsin or UCL.
 
 
\subsubsection{Brief description of the activities of the tech transfer office}
 About SATT LUTECH : SATT LUTECH is a privately owned company specialized in transfer and commercialization of innovative technologies. The company was created by Université Pierre et Marie Curie, CNRS, Université de Technologie de Compiègne, Muséum national d’Histoire naturelle, INSEAD, Université Panthéon-Assas, Ecole Nationale Supérieure de design et de Création Industrielle – and the Caisse des Dépôts group which is a "public group serving general interest and economic development". SATT LUTECH was created on January 31, 2012. Following its successful bid in the French government’s “Investing in the Future” program, SATT LUTECH was awarded 20 million euros for its first three years of operations, and will amount 73 million euros over ten years. SATT LUTECH’s role is to focus on the transfer and commercialization of technologies issued from the research laboratories of its shareholders.

SATT LUTECH covers three main activities:
\begin{itemize}
\item Detection of research results developed in its shareholders laboratories which could lead to commercial innovations.
\item Investment in the development of research results and demonstration of their potential through pilot projects, on a scale and under conditions that will create interest of companies and / or investors.
\item Commercialization of matured technologies to an existing company or through the creation of a start-up.
\end{itemize}

SATT LUTECH invests in the following areas:
\begin{itemize}
\item Health
\item Information technology and communication
\item Chemistry - Materials – Processes
\item Environment and Energy
\end{itemize} 

\subsubsection{Initiatives of the tech transfer office}
Success stories in Medical/Pharmaceutical:
\begin{itemize}
\item HIV Detection Kit: extensive licensing, revenues currently at 800 k€/year
\item Cellectis: publically listed, 2010 expected turnover: 24.7 M€ (+100\%)
\item CARMAT: publically listed, founded in 2008, and 40 M€ invested
\item Supersonic Imagine: founded in 2005, 61,5 M€ invested
\item Fovea: acquired by Sanofi for 370 M€
\end{itemize}
Information Technology
\begin{itemize}
\item Qosmos: founded in 2000, turnover in 2010 was 9.3 M€—a 40\% increase from 2009, and ten times the turnover in 2005
\item Sensitive Objects: acquired by Tyco in 2010 for 44 M€2
\end{itemize}
 
\subsubsection{LUTECH Research Potential}
Cutting-Edge Researchers
\begin{itemize}
\item 5 Fields Medal laureates
\item 44 European Research Council Grants
\item More than 60 members of the French Academy of Science
\end{itemize}
Broad Range of Disciplines organised in 5 themes:
\begin{itemize}
\item Computer Science, Mathematics, Engineering
\item Material Sciences
\item Environment and Earth Science
\item Life and Health Sciences
\item Arts, Humanities, Social and Organisation Sciences
\end{itemize}
 
\subsubsection{List of companies which collaborate with the institution}
\begin{itemize}
\item  Shareholders :
\item CNRS
\item INSEAD
\item Université Pierre et Marie Curie
\item Université Panthéon-Assas
\item Université de Technologie de Compiègne
\item Caisse des Dépôts et Consignations
\item Museum national d’Histoire naturelle
\item Les Ateliers-Paris Design Institute (École Nationale Supérieure de Création Industrielle)
\end{itemize}
 
\subsection{The JSI technological transfer facility}

\subsubsection{Name of the office for tech transfer}
Center for Tehnology Transfer and Innovation, Jožef Stefan Institute, Ljubljana. Head of unit: dr. Špela Stres, LLM, patent attorney 
\subsubsection{Number of people involved in the activity}

11 (dr. Spela Stres, dr. Levin Pal, dr. Marija Nika Lovšin, dr. Urban Odić, mag. Robert Blatnik, mag. Marjeta Trobec, France Podobnik, Urban Šegedin, Alen Draganović, Lea Kane, Miha Goriup), contact details available: \url{http://tehnologije.ijs.si/ttwiki/en/Workers}

\subsubsection{Brief description of the activities of the tech transfer office}
The Centre for technology and innovation (CTT) at the "Jozef Stefan" Institute operates as an independent centre within the Institute since 2010. The JSI is the largest Slovenian institute for research in science, engineering sciences and environmental sciences. Their mission is to connect science with the society: science with business and science with education. They assist in organizing and carrying out contract research and other collaborations with industry, licensing and spin-offing and at individual technology projects of the Institute.

\subsection{The UB technological transfer facility}

%%\bibliographystyle{alpha}
%%\bibliography{main-bib}

\end{document}

%%% Local Variables:
%%% mode: latex
%%% TeX-master: t
%%% save-place: t
%%% End:
