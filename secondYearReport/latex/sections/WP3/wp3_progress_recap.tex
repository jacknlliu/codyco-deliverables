%!TEX root = ../../secondYearReport.tex


 
\paragraph*{WP3: control and optimization of whole-body motion in contact (UPMC)}

After two years of project, the level of achievement of the objectives in WP3 meets the expectations. The main achievements are:
\begin{itemize}
\item[T3.3] \textbf{Studies} by UPMC and UB on the extension of whole-body control frameworks in order to account for non-rigid contacts.  The approach retained by UPMC adapts the desired contact force value and center of mass trajectory in order to establish stable and supporting contacts as fast as possible. This approach assumes a local contact model but does not require the explicit knowledge of its parameters and rather uses the velocity of the contact point to directly adapt the desired contact force. UB computes optimal contact forces based on the momentum of the robot and computes and converts this forces into desired acceleration using an estimated model of the surface in contact. Given these desired accelerations, joint torques are computed in order to best achieve the desired accelerations.
\item[T3.4] \textbf{Integration} by IIT and UPMC of the whole-body controllers on the real robots. This includes a large amount of background work by IIT on low-level control aspects (calibration and identification mostly).
\item[T3.4] \textbf{Investigation} by UPMC of MPC as an efficient mean to optimally handle the postural balancing problem under varying contact conditions.
\item[T3.4] \textbf{Studies} by Inria, TUD and UPMC on the adaptation of task weights in the whole-body controller.
\item[T3.4] \textbf{Studies} by TUD on learning inverse dynamics with contacts to predict contact forces.
\item[T3.4] \textbf{Studies} by UB on projected operational space dynamics for the control of constrained motion for a manipulator performing a task while in contact with the environment.
\end{itemize}



