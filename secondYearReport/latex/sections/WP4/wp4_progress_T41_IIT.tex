%!TEX root = ../../secondYearReport.tex


Within T4.1 IIT developed a theoretical framework for estimating whole-body
dynamics from distributed multimodal sensors \cite{Nori2015}. Considered sensors
include joint encoders, gyroscopes, accelerometers and force/torque sensors.
Estimated quantities are position, velocity, acceleration and (internal and
external) wrenches on all the rigid bodies composing the robot articulated
chain. The estimation procedure consists of an extended Kalman filter (EKF)
which gives the a-posteriori estimation given all the available measurements.
Computational efficiency is obtained by formulating the Kalman filter
update-step with a sparse Bayesian network. Experiments for validating the
proposed theoretical framework have been conducted on a leg of the iCub humanoid
robot. The iCub is an ideal platform for the proposed experiment given its
distributed force, torque, linear acceleration and angular velocity sensors.
Results have shown the accuracy and the computational efficiency of the proposed
method. The theoretical framework has been implemented in an open source
software (see also Section \ref{sec:T15}).