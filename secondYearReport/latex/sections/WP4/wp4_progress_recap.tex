%!TEX root = ../../secondYearReport.tex


\paragraph*{WP4: adaptation, generalization and improvement of compliant control and tasks with contacts (TUD)}

The goal of WP4 is to endow the CoDyCo humanoid robot control architecture with
the core abilities for the adaptation, generalization and self-improvement of
both control laws and tasks that involve physical interaction with humans, and
the environment.

%withot inv dyn learning (inv dyn learning is part of WP3)
During the second year, IIT developed a theoretical framework for estimating whole-body
dynamics from distributed multimodal sensors \cite{Nori2015}.
TUD continued their research in probabilistic movement
primitive representations. A journal article on imitation learning and the
co-activation of basic skills is under review. For multi-modal solutions an
extension using mixtures of Gaussians with latent variables was published at an
international robotics conference \cite{Rueckert_2015}. In tasks with
contacts, a model-free probabilistic movement representation was developed that
models joint distributions over kinematic and force trajectories. This work is
under review. Further, TUD investigated noise robust planning methods methods to
plan movement skills given task-space constraints. This work was published at an
international humanoid robot conference \cite{Rueckert2014} and will be
used in year three for generating optimal control laws with learned dynamics
models. During year two, TUD investigated the learning of temporal activation
profiles of low-level task controller. This work is under review. 
A similar line of research was conducted by UPMC who studied how to deal with interferences in task combinations in whole-body.
The tasks are parameterized with Dynamical Movement Primitives, whose parameters are optimized
based on a general compatibility principle. This work was published at an 
international humanoid robot conference \cite{lober-HUMANOIDS2014}. 
In a second study that is currently under review, UPMC studied how task
variability can be used to modulate task priorities during their execution. 
UB continued their research on computed torque control leveraging low-gain control. 