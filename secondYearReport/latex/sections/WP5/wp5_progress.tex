%!TEX root = ../../secondYearReport.tex

\paragraph{Work package 5 progress}

The activities in WP5 are divided into four tasks corresponding to the four years project duration. As a result, during the second year CoDyCo results concentrate on T5.2. The main result consist in the implementation of the validation scenario consisting of the balancing on different type of rigid contacts.

\subparagraph{Scenario 2: iCub posture control while performing goal directed actions (T5.2)}

The main contributions to T5.2 have been presented in ``Validation scenario2: balancing on feet while performing goal directed actions.'' which discusses the technical implementation of the second year validation scenario (see \url{https://github.com/robotology-playground/codyco-deliverables/tree/master/D5.2/pdf}). The software developed for the scenario implementation is released with an open-source license and distributed through github (\url{https://github.com/robotology/codyco-modules} ). The main software activities include: a module to identify the whole-body motor transfer functions (\url{https://github.com/robotology/codyco-modules/tree/master/src/modules/motorFrictionIdentification}), a module for estimating whole-body internal (joint torques) and external (contact) forces (\url{https://github.com/robotology/codyco-modules/tree/master/src/modules/wholeBodyDynamicsTree}), a module for whole-body joint torque control (\url{https://github.com/robotology/codyco-modules/tree/master/src/devices/jointTorqueControl}), a C++ module for whole-body control under multiple rigid contacts (\url{https://github.com/robotology/codyco-modules/tree/master/src/modules/torqueBalancing}).

\subparagraph{Deviations from workplan}  

The original work plan was leaving quite a flexible set of possibilities for both the postural task (e.g. sitting on a chair or balancing on the feet) and the goal oriented action (e.g. opening a drawer while standing or manipulating object while sitting). In the final validation scenario it was chosen to consider a interactive scenario, with the torque controlled iCub standing on his feet while trying to grasp an object moved by the experimenter. As soon as the object exceeds the robot workspace, the iCub takes a forward step in order to increase his workspace. 

%\begin{itemize}
%\item[-] \emph{\color{red}[A summary of progress towards objectives and details for each task;]}
%\item[-] \emph{\color{red}[Highlight clearly significant results;]}
%\item[-] \emph{\color{red}[If applicable, explain the reasons for deviations from Annex I and their impact on other tasks as well as on available resources and planning;]}
%\item[-] \emph{\color{red}[If applicable, explain the reasons for failing to achieve critical objectives and/or not being on schedule and explain the impact on other tasks as well as on available resources and planning (the explanations should be consistent with the declaration by the project coordinator) ;]}
%\item[-] \emph{\color{red}[a statement on the use of resources, in particular highlighting and explaining deviations between actual and planned  person-months per work package and per beneficiary in Annex 1 (Description of Work);]}
%\item[-] \emph{\color{red}[If applicable, propose corrective actions.]}
%\end{itemize}
