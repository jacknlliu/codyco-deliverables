%!TEX root = ../../secondYearReport.tex

\paragraph{Work package 5 progress}

The activities in WP5 are divided into four tasks corresponding to the four years project duration. As a result, during the second year CoDyCo results concentrate on T5.1. The main result consist in the implementation of the validation scenario consisting of the balancing on different type of rigid contacts.

\subparagraph{Scenario 1: iCub balancing on multiple rigid contacts (T5.1)}

The main contributions to T5.1 have been presented in ``D5.1 Scientific report on validation scenario 1: balancing on multiple rigid contact points.'' which discusses the technical implementation of the second year validation scenario (see \url{https://github.com/robotology-playground/codyco-deliverables/tree/master/D5.1/pdf}). The software developed for the scenario implementation is released with an open-source license and distributed through github (\url{https://github.com/robotology/codyco} ). The main software activities include: a module to identify the whole-body motor transfer functions (\url{https://github.com/robotology/codyco/tree/master/src/modules/motorFrictionIdentification}), a module for estimating whole-body internal (joint torques) and external (contact) forces (\url{https://github.com/robotology/codyco/tree/master/src/modules/motorFrictionIdentification}), a module for whole-body joint torque control (\url{https://github.com/robotology/codyco/tree/master/src/modules/jointTorqueControl}), a C++ library that implements the wholeBodyInterface in simulink (\url{https://github.com/robotology/codyco/tree/master/src/simulink}).

\subparagraph{Deviations from workplan}  

The original work plan have foreseen contacts at feet, hands, back, buttocks, arms and legs. The final validation scenario will only include possible contacts at hands and feet. This simplification is mainly due to the fact that at he end of the CoDyCo second year the iCub does not yet include tactile sensing on the back, legs and buttocks. These sensors will be soon included in the iCub and the CoDyCo software is already designed to include this information. 

%\begin{itemize}
%\item[-] \emph{\color{red}[A summary of progress towards objectives and details for each task;]}
%\item[-] \emph{\color{red}[Highlight clearly significant results;]}
%\item[-] \emph{\color{red}[If applicable, explain the reasons for deviations from Annex I and their impact on other tasks as well as on available resources and planning;]}
%\item[-] \emph{\color{red}[If applicable, explain the reasons for failing to achieve critical objectives and/or not being on schedule and explain the impact on other tasks as well as on available resources and planning (the explanations should be consistent with the declaration by the project coordinator) ;]}
%\item[-] \emph{\color{red}[a statement on the use of resources, in particular highlighting and explaining deviations between actual and planned  person-months per work package and per beneficiary in Annex 1 (Description of Work);]}
%\item[-] \emph{\color{red}[If applicable, propose corrective actions.]}
%\end{itemize}
