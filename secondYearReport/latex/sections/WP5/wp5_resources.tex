%!TEX root = ../../secondYearReport.tex

\subparagraph{Resources}

Resources were used with small difference with respect to what planned. In particular IIT invested only 2 PM with respect to 12PM planned. The motivation resides in the fact that WP5 took advantage of the significant effort done in WP1 (software) and WP3 (control) and in a sense resources initially planned on T5.1 eventually have been committed to T1.1, T1.2, T1.3, T3.1 and T3.2.

\begin{center}
\begin{tabular}{|C{1.5cm}|C{1.5cm}|C{1.5cm}|C{2cm}|C{2cm}|C{2cm}|C{2cm}|}
\hline
\footnotesize \textbf{WP5 person months}& \footnotesize \textbf{IIT}&\footnotesize \textbf{TUD}&\footnotesize \textbf{UPMC}& \footnotesize \textbf{UB} &\footnotesize \textbf{JSI} & \footnotesize \textbf{INRIA} \\ \hline
\footnotesize Year 1 &  2.00 & 0.00 & 0.31 & 0.00 & 0.00 & 0.00     \\  \hline
\footnotesize Year 2 &  12.00 & 0.85 & 0.05 & 0.00 & 0.00 & 0.00     \\  \hline
\footnotesize Partial &  14.00 & 0.85 & 0.36 & 0.00 & 0.00 & 1.50 \\ \hline \hline
\footnotesize Overall &  48.00 & 5.00 & 2.50 & 0.00 & 0.00 & 1.50 \\ \hline
\end{tabular}
\end{center}

\subparagraph{Deviations from workplan} 
The original work plan was leaving quite a flexible set of possibilities for both the postural task (e.g. sitting on a chair or balancing on the feet) and the goal oriented action (e.g. opening a drawer while standing or manipulating object while sitting). In the final validation scenario it was chosen to consider a interactive scenario, with the torque controlled iCub standing on his feet while trying to grasp an object moved by the experimenter. As soon as the object exceeds the robot workspace, the iCub takes a forward step in order to increase his workspace. 

UPMC covered the activities within WP5 with own resources for a total of 0.45 PM. In particular Ryan Lober (PhD student, French PhD research grant) and Darwin Lau (Postdoctoral scholar) participated to the deployment of the whole-body interface developed in WP1 on the iCub robots present at UPMC.
