%!TEX root = ../../thirdYearReport.tex

Within T4.1 IIT developed in the first and the second year of the 
project a theoretical framework for estimating whole-body
dynamics from distributed multimodal sensors \cite{nori2015}. Considered sensors
include joint encoders, gyroscopes, accelerometers and force/torque sensors.
Estimated quantities are position, velocity, acceleration and (internal and
external) wrenches on all the rigid bodies composing the robot articulated
chain. In the third year of the project, IIT investigated the integration
of this estimation techniques with the classical identification techniques
for inertial parameters that were implementer in software packages as part of T1.4 .

In \cite{nori2015}, the estimation of dynamics quantities was performed by using 
the uncertancy of each sensor  was learned using real data.
In particular the Expectation-Maximization algorithm was used to estimate the covariance matrix of 
each sensor. IIT focused on extending this EM algorithm to also learn the mass,
center of mass and inertia matrix of each link of the robot.
This extension was validated in simulation and submitted in \cite{traversaro2015parametersEM}. 
 

