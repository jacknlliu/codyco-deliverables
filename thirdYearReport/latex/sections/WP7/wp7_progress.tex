%!TEX root = ../../secondYearReport.tex


\paragraph{Work package 7 progress}

Dissemination and exploitation activities included the participation to international events addressed to both commercial and academic institutions. 

\subparagraph{Dissemination activities towards academia, industry, and other users (T7.1)}

Dissemination activities were conducted thorough international publications, organisation of international events, talks at international conferences, press interviews and iCub expositions at international events. Here is the overall contribution subdivided by partner:

\begin{itemize}

\item IIT: 4 invited talks, 4 organised international events, 6 talks at international conferences, 10 publications (2 journal, 7 internal conferences, 1 book chapter), 10 media coverage events.

\item TUD: 15 invited talks, 4 organised international events, 7 publications (7 internal conferences), 7 media coverage events.

\item UPMC: 3 invited talks, 6 publications (6 internal conferences), 1 media coverage event.

\item UB: 3 invited talks, 5 publications (5 internal conferences), 5 talks at international conferences.

\item JSI: 1 invited talks, 1 organised special issue, 2 talks at international conferences, 7 publications (1 journal, 6 internal conferences).

\item INRIA: 4 invited talks, 4 organised international events, 8 publications (4 journal, 4 internal conferences), 3 media coverage events

\end{itemize}

Live demonstration of the iCub have been performed at several international events.  Some of these events were sponsored by CoDyCo and the following is a non exhaustive list:

\begin{enumerate}

\item 12$^{th}$-14$^{th}$ March 2014. EU Robotics Forum Rovereto. \url{http://www.erf2014.eu/erf_home.jsp}.
\item 3$^{rd}$-6$^{th}$ June 2014. Automatica 2014, Munich, Germany. \url{http://www.nfm-automatica.de/2014/en/home.php}.
\item 3$^{rd}$-5$^{th}$ October 2014. European Maker Faire, Roma, Italy. \url{http://www.makerfairerome.eu/en/agenda2014/}. 
\item 18$^{th}$-20$^{th}$ November 2014. 2014 IEEE-RAS International Conference on Humanoid Robots (Humanoids 2014), Madrid, Spain. \url{http://www.humanoids2014.com}.

\end{enumerate} 

Among the invitations as a speaker at international events it is worth citing the following:

\begin{enumerate}

\item Francesco Nori: invited speaker at the Journ�es Nationales du GdR Robotique 2014, held at Grand amphith\'e\^atre du Centre Arts et M\`etiers ParisTech, 151-155 boulevard de l'H�pital, 75013 Paris. 30 October 2014. \url{http://www.gdr-rob2014.org}.

\item Serena Ivaldi: invited speaker French-German-Japan Workshop on Humanoid and Legged robots. Social learning and engagement in human-humanoid interactions. \url{http://orb.iwr.uni-heidelberg.de/hlr2014/HLR14}.

\item Jan Babic: invited talk in Paris at the Universit� Pierre et Marie Curie. Synthesis of skilled robotic behaviour through human sensorimotor adaptation:  12$^th$ November 2014.

\item Jan Peters: keynote for the Learning by Demonstration Session Topic at the IEEE/RSJ International Conference on Intelligent Robots and Systems (IROS), Chicago, USA.

\item Jan Peters: invited plenary talk speaker at the 13th International Conference on Intelligent Autonomous Systems (IAS-13), Padua, Italy.

\item Vincent Padois: invited talk at the Cap Digital/Innorobo day about Robotics et Innovations. ``Issues and challenges of interactive robotics in complex industrial contexts''. Lyon, France - March 2014.

\item Michael Mistry: invited Lecturer at European Computational Motor Control Summer School. June 15$^th$-21$^st$, 2014.

\end{enumerate} 

Among the organised international events here is a non exhaustive list of the most relevant events:

\begin{enumerate}

\item IIT: workshop organisation at the 2014 IEEE-RAS international conference on humanoid robots (Humanoids 2014). ?One day with a humanoid robot: a crash course on the iCub software tools?. Coordinators: L. Natale, F.Nori, U. Pattacini, V. Tikhanoff, M. Randazzo, G. Metta (Italy).

\item IIT: iCub summer school (Veni Vidi Vici 2014). In 2014, the school was held in Sestri Levante, Italy, July 21-30 2014. Main organisers: Giorgio Metta, Lorenzo Natale, Francesco  Nori, Vadim Tikhanoff, Ugo Pattacini.

\item INRIA, TUD, UB, JSI: editors for the Autonomous Robots special Issue: ``Whole-body control of contacts and dynamics for humanoid robots''. 

\end{enumerate} 



\subparagraph{Exploitation plan (T7.2)}

The second year activities on T7.1 and T7.2 are all contained in ``D7.1 Dissemination and exploitation plan'' available here: \url{https://github.com/robotology-playground/codyco-deliverables/tree/master/D7.1/pdf}.

\subparagraph{Management of IPR (T7.3)}

No activities to be reported during the second year on this task in consideration of the fact that the task started at the very end of the second year. As a minor starting activity the consortium circulated a list containing each partner responsible contact person for the IPR management. This list is contained in ``D7.1 Dissemination and exploitation plan'' available here: \url{https://github.com/robotology-playground/codyco-deliverables/tree/master/D7.1/pdf}.

\subparagraph{Dissemination of a database of human motion with contacts (T7.4)}

During the second year of CoDyCo, IIT completed the task of setting up a database for storing both human and robot datasets. The details on the database are reported in ``D7.2 Standard database with support materials'' available here \url{https://github.com/robotology-playground/codyco-deliverables/tree/master/D7.2/pdf}. 

