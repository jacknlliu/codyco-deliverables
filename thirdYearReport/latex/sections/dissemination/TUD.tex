%!TEX root = ../../thirdYearReport.tex
\subsection{TUD contributions to dissemination}

9 invited talks, 1 organised international events, 12 publications (2 journal articles, 10 international conferences), 3 media coverage events, 3 M.Sc. theses and one Ph.D. thesis. 

\subsubsection{Invited talks}

\begin{enumerate}
%year three (03.2015-02.2016)

% Roberto
\item Talk by Roberto Calandra, 16 Oct 2015 University College London, London, UK, host: Guy Lever.
\item Talk by Roberto Calandra, 14 Oct 2015 University of Oxford, Oxford, UK, host: Michael Osborne, Machine Learning Research Group.
\item Talk by Roberto Calandra, 13 Oct 2015 Imperial College London, London, UK, host: Stefan Leutenegger, Dyson Robotics Lab.
\item Talk by Roberto Calandra, 03 Jun 2015 University of British Columbia, Vancouver, Canada, host: Mark Schmidt.
\item Talk by Roberto Calandra, 02 Jun 2015 University of Washington, Seattle, US, host: Dieter Fox, Robotics and State Estimation Lab.
\item Talk by Roberto Calandra, 01 Apr 2015 TU Freiburg, Freiburg, Germany, host: Frank Hutter.

%Elmar
\item Talk by Elmar Rueckert, 11/2015 Understanding Human Motor Control through Robotics Applications. Invited Talk in Prof. Constantin Rothkopf’s seminar on ’research and applications of psychology in IT’, Darmstadt, Germany.
\item Talk by Elmar Rueckert, 02/2015 Probabilistic Inference and Modeling of Human Motor Skill Learning. Invited Talk. Workshop with Marc Toussaint’s group, Wolfram Burgard’s group and Oliver Brock’s group, Manigod, France.

%Jan
\item Talk by Jan Peters, 06/2015 Universit\"at Ulm, Host: F.~Kargl, Ulm, Germany, July, 2015.
% \item Talk by Jan Peters, 9/17/2014 IEEE/RSJ International Conference on Intelligent Robots and Systems (IROS), Session [1] Keynote for the Learning by Demonstration Session Topic, Chicago, USA
% \item Talk by Jan Peters, 7/16/2014 13th International Conference on Intelligent Autonomous Systems (IAS-13), Padua, Italy 
% \item Talk by Jan Peters, 7/11/2014 British Machine Vision Association (Vision for language and manipulation), London, UK
% \item Talk by Jan Peters, 7/9/2014 IEEE/ASME Conference on Advanced Intelligent Mechatronics (AIM2014), Besancon, France
% \item Talk by Jan Peters, 3/27/2015 RSS 2015 Symposium: Frontiers of Robotics, New Brunswick, USA 
% \item Talk by Jan Peters, 9/14/2014 IROS Workshop: Machine Learning in Planning and Control of Robot Motion, Chicago, USA
% \item Talk by Jan Peters, 6/5/2014 ICRA Workshop: iCub and Friends, Hong Kong, China
% \item Talk by Jan Peters, 2/6/2015 Carnegie Mellon University, Robotics Institute Lecture Series , Host: S. Srinivasa, J.A. [55] Bagnell, Pittsburgh, PA, USA
% \item Talk by Jan Peters, 1/16/2015 Technische Universität München, Host: E. Steinbach, Munich, Germany [56]
% \item Talk by Jan Peters, 12/4/2014	KIT, Host: W. Juling, Karlsruhe, Germany [57]
% \item Talk by Jan Peters, 11/28/2014	ETH Zürich, ETH Distinguished Lecture Series, Host: R. Siegwardt, Zürich, Switzerland
% \item Talk by Jan Peters, 7/7/2014 Universität Hamburg, Host: U. von Luxburg, Hamburg, Germany [59]
% \item Talk by Jan Peters, 5/9/2014 Université de Liege, Host: D. Ernst, Liege, Belgium [60]
% \item Talk by Jan Peters, 2/5/2014 ABB AG, Corporate Research Center, Robotics and Manufacturing, Host: H. Ding, Ladenburg, Germany
% \item Talk by Jan Peters, 2/4/2014 Technical University of Graz, Host: W. Maas, Graz,Austria [62]
\end{enumerate}

\subsubsection{Pubblications}

\begin{enumerate}
%year three (03.2015-02.2016)
%journals
\item E Rueckert, D Kappel, D Tanneberg, D Pecevski and J Peters. Recurrent Spiking Networks Solve Planning Tasks. Scientific Reports, Nature Publishing Group, 2016.
\item R Calandra, A Seyfarth, J Peters and M P Deisenroth. Bayesian optimization for learning gaits under uncertainty. Annals of Mathematics and Artificial Intelligence, pages 1–19, 2015.
%conferences
\item J Kohlschuetter, J Peters and E Rueckert. Learning Probabilistic Features from EMG Data for Predicting Knee Abnormalities. In Proceedings of the XIV Mediterranean Conference on Medical and Biological Engineering and Computing (MEDICON), 2016.
\item V Modugno, G Neumann, E Rueckert, G Oriolo, J Peters and S Ivaldi. Learning soft task priorities for control of redundant robots. In Proceedings of the International Conference on Robotics and Automation (ICRA), 2016.
\item R Calandra, S Ivaldi, M Deisenroth, E Rueckert and J Peters. Learning Inverse Dynamics Models with Contacts. In Proceedings of the International Conference on Robotics and Automation (ICRA). 2015.
\item R. Calandra, S. Ivaldi, Marc. P. Deisenroth, E. Rueckert, and J. Peters. Learning inverse dynamics models with contacts using tactile sensors. ICRA 2015 Workshop on Tactile \& force sensing for autonomous, compliant, intelligent robots, 2015.
\item E Rueckert, J Mundo, A Paraschos, J Peters and G Neumann. Extracting Low-Dimensional Control Variables for Movement Primitives. In Proceedings of the International Conference on Robotics and Automation (ICRA). 2015. 
\item S Traversaro, A Del Prete, S Ivaldi and F Nori. Avoiding to rely on Inertial Parameters in Estimating Joint Torques with proximal F/T sensing. In Proceedings of the International Conference on Robotics and Automation (ICRA). 2015.
\item A Paraschos, E Rueckert, J Peters and G Neumann. Model-free Probabilistic Movement Primitives for physical interaction. In Intelligent Robots and Systems (IROS), 2015 IEEE/RSJ International Conference on. 2015, 2860–2866. 
\item E Rueckert, R Lioutikov, R Calandra, M Schmidt, P Beckerle and J Peters. Low-cost Sensor Glove with Force Feedback for Learning from Demonstrations using Probabilistic Trajectory Representations. In ICRA 2015 Workshop on Tactile and force sensing for autonomous compliant intelligent robots. 2015.
\item L Fritsche, F Unverzag, J Peters and R Calandra. First-person tele-operation of a humanoid robot. In Humanoid Robots (Humanoids), 2015 IEEE-RAS 15th International Conference on. 2015, 997–1002.
\item R Calandra, S Ivaldi, M P Deisenroth and J Peters. Learning torque control in presence of contacts using tactile sensing from robot skin. In Humanoid Robots (Humanoids), 2015 IEEE-RAS 15th International Conference on. 2015, 690–695.
\end{enumerate}
	
\subsubsection{Media coverage}

\begin{enumerate}
%Elmar
\item 09/2015 Organized by Elmar Rueckert, Kinderuni Darmstadt. Interactive robot demonstrations of the Nao, the iCub and the Darias robots. Supported by Veronika Weber and Guilherme J. Maeda.
\item 04/2015 Interview of Jan Peters, Major German TV program, SAT1. Life demonstrations of teaching the iCub how to stack cup.
\item 03/2015 Organized by Elmar Rueckert, KID Science Radioclub. Lab tour and life demonstrations of the Oncilla, the iCub and the Darias robots. Supported by Veronika Weber, Guilherme J. Maeda, Rudolf Lioutikov and Roberto Calandra.

%Jan, list below is from year two
% \item http://www.1730live.de/intelligente-maschinen/
% \item 2/17/2015 Focus, Müssen wir uns bald vor intelligenten Systemen fürchten?, Germany
% \item 10/21/2014 Wired (German Edition), Aus Fails lernen: Roboter der TU-Darmstadt optimieren sich autonom,
% \item 2/12/2015 Frankfurter Rundschau, Menschliche Maschinen, Germany, by Franziska Schubert. 
% \item 12/1/2014 Hoch3, Ich kann täglich mehr, Germany.
% \item 9/17/2014 The Guardian, Win tickets to see what tomorrow?s world may hold, UK.
% \item 28/3/2014 DRadio Wissen, Tischtennisroboter: Falscher Aufschlag, Germany.
\end{enumerate}

\subsubsection{MSc. and Ph.D. theses}%year three (03.2015-02.2016)
\begin{enumerate}
 \item Stark S. MSc. thesis. Learning Probabilistic Feedforward and Feedback Policies for Generating Stable Walking Behaviors. 2016.
 \item Kohlschuetter J. MSc. thesis. Learning Probabilistic Classifiers from Electromyography Data for Predicting Knee Abnormalities. 2016.
 \item D Tanneberg. MSc. thesis. Spiking Neural Networks Solve Robot Planning Problems. 2016.
 \item O Kroemer. Machine Learning for Robot Grasping and Manipulation. 2015.
\end{enumerate}

\subsubsection{Student research stays}

\begin{enumerate}
 \item E Rueckert, 2014. Jozef Stefan Institute, Slovenia, Department of Automation, Biocybernetics and Robotics, Prof. Dr. Jan Babic. Research internship on investigating the functional role of supportive contacts in human postural control. 
\end{enumerate}

\subsubsection{Organised conference workshops}

\begin{enumerate}
\item  R.~Calandra: Organizer of the Workshop on Bayesian Optimization (BayesOpt) at NIPS 2015. Web: http://bayesopt.github.io/
\end{enumerate}

%here is our list from the last year
% \begin{enumerate}
% \item  ICRA 2015: organizer of Workshop "Tactile and force sensing for autonomous, compliant and intelligent robots?. Web: http://www.ausy.tu-darmstadt.de/Workshops/ICRA2015TactileForce 
% \end{enumerate}

%\subsubsection{Meetings with industrial partners}

%TACMAN List
%\begin{enumerate}
%*** TUDA
%\item  H.~van Hoof, ``Research visit'', \emph{SynTouch}, Los Angeles, USA, May, 2015.
%\item  F.~Veiga, ``Research visit'', \emph{DLR}, Oberpfaffenhofen, Germany, December, 2015.
%\end{enumerate}

\subsubsection{Invited speakers}%year three (03.2015-02.2016)
\begin{enumerate}
	\item W.~Kellermann, Invited Speaker at TUDA, \emph{Friedrich-Alexander Univerisit\"at Erlangen-N\"urnberg}, Erlangen, Germany, December, 2015.
	\item D.~Nikolic, Invited Speaker at TUDA, \emph{Max-Planck Institut for Brain Research}, Frankfurt, Germany, January, 2016.
	\item V.~Lippi, Invited Speaker at TUDA, \emph{Uniklinik Freiburg}, Freiburg, Germany, February, 2016.
	\item F.~Hutter, Invited Speaker at TUDA, \emph{Universit\"at Freiburg}, Freiburg, Germany, February, 2016.
\end{enumerate}

\subsubsection{Collaborations}

%ACTIVE collaborations where TUD is involved.
\begin{enumerate}%Note: names are listed in alphabetical order
	\item UB and TUD, involved are M.~Azad, M.~Mistry, J.~Peters, E.~Rueckert. Title: \emph{Uncertainty in contact}, first results on TUD's iCub. Paper submission planned for a robotics conference (HUMANOIDS, ICRA) in 2016. 
	\item TUD and JSI, involved are J.~Babic, J.~Camernik, J.~Peters, E.~Rueckert. Title: \emph{Postural control predicts volitional motor control}, paper submitted for review at Scientific Reports, 01/2016.
\end{enumerate}